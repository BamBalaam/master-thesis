\chapter{Table of contents draft}
\label{chap:draft_toc}

\begin{enumerate}
    %\setlength\itemsep{-0.4em}
    \item \textbf{Introduction} (see Chapter 1 for a first draft of this introductory chapter)
        \vspace{-0.4cm}
        \begin{enumerate}[label*=\arabic*.]
            \setlength\itemsep{-0.5em}
            \item \textbf{Universal language}
            \item \textbf{Constructed languages}
            \item \textbf{Lojban}
            \vspace{-0.4cm}
            \begin{enumerate}[label*=\arabic*.]
                \setlength\itemsep{-0.5em}
                \item \textbf{The roots: the history of Loglan}
                \item \textbf{The creation of Lojban}
                \item \textbf{Brief introduction to the Lojban grammar}
            \end{enumerate}
            \item \textbf{What this thesis aims to achieve}
        \end{enumerate}
    \item \textbf{State of the art}
        \vspace{-0.4cm}
        \begin{enumerate}[label*=\arabic*.]
            %\setlength\itemsep{-0.5em}
            \item \textbf{Existing research} (brief analysis of existing research on language parsers in general, and Lojban specifically)
            \item \textbf{Existing resources} (brief overview of tools, resources, and software that exists around the language)
        \end{enumerate}
    \item \textbf{Longer introduction to the Lojban grammar} (required to understand the following chapters and how the parser will be constructed)
    \item \textbf{Parsing of Lojban sentences}
        \vspace{-0.4cm}
        \begin{enumerate}[label*=\arabic*.]
            %\setlength\itemsep{-0.5em}
            \item \textbf{Description of the created library} (writing code which parses a Lojban sentence and creates, using formal logic, a framework to understand the sentence based on the Lojban grammar)
            \item \textbf{Examples} (showing examples of the generated logical framework and how it effectively describes the Lojban sentence)
        \end{enumerate}
    \item \textbf{Deducing meaning from a parsed sentence}
        \vspace{-0.4cm}
        \begin{enumerate}[label*=\arabic*.]
            %\setlength\itemsep{-0.5em}
            \item \textbf{Description of the created library} (drawing upon the code written and outlined in the previous chapter, attempting to translate the Lojban sentence into a sensible sentence in English using a dictionary)
            \item \textbf{Examples} (showing examples of translations achieved with the program written)
        \end{enumerate}
    \item \textbf{Using Lojban as an interlingua} (small attempts, with a limited corpus, of creating a system of basic machine translation using Lojban as an interlingua)
    \item \textbf{Conclusions}
    \item \textbf{Bibliography}
    \item \textbf{Annex A - Predicate Logic} (small annex giving an introduction to predicate logic, in case the readers need a refresher to understand the basis of the Lojban grammar)
\end{enumerate}
