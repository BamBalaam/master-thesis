\section*{Résumé} % (fold)
\label{sec:resume}

\subsection*{Informations}

\begin{itemize}
    \setlength\itemsep{0.1em}
    \item \textbf{Nom et prénom:} MADEIRA CORTES André
    \item \textbf{Filière:} Master en Sciences et Technologies de l'Information et de la Communication
    \item \textbf{Année académique:} 2023-2024
    \item \textbf{Titre du mémoire:} Language processing of a constructed language: the case of Lojban.
\end{itemize}

\subsection*{Mots-clés}

lojban, langue construite, apprentissage de langues construites, python, traitement automatique des langues

\subsection*{Brève description du mémoire}

Ce mémoire explore l'intersection entre deux champs d'étude: la linguistique, via la découverte d'une langue
construite appelée "Lojban", et l'informatique, via la thématique du traitement automatique des langues.
Il a deux objectifs distincts, l'un pratique et l'autre théorique:
L'objectif pratique est de créer une série d'outils, écrits avec le language de programmation Python,
qui permettent d'analyser syntaxiquement une phrase écrite en Lojban et d'en visualiser les parties composantes.
L'objectif théorique est de créer une version simplifiée de grammaire d'analyse syntaxique pour la langue Lojban qui permet,
bien sur, de "parser" des phrases, mais également d'étudier la grammaire de façon plus ludique.
Les résultats de ces deux objectifs sont évalués par l'exécution d'une batterie de tests qui évalue si un jeu de données
composé de phrases extraites du livre de référence pour l'apprentissage du Lojban sont validées correctement.