\chapter{Tagging and labeling parsed sentences: the ``visitor'' module}
\label{chap:visitor-module}

\section{Methodology}

Parsimonious has the visitor function. Allows to explore the parse tree generated

What we achieve is a mix of Parts-Of-Speech tagging (giving each word a grammatical category) and Semantic role labeling
because tagging verbs allows us, thanks to their definition, to know which word accomplishes what role in the sentence.

% https://www.geeksforgeeks.org/nlp-part-of-speech-default-tagging/

% https://en.wikipedia.org/wiki/Semantic_role_labeling

\section{Breakdown of the module}
\label{sec:visitor-code-breakdown}

The code produced is found at Annex \ref{appendix:gentufa-visitor-annex}.

TODO

\newpage

\section{Examples of usage}

This module can be used in two different ways: as a command-line interface (CLI) tool
from a terminal, or as a class for short scripts or other modules to use.

\begin{lstlisting}[caption=GentufaVisitor module being used as a command-line interface tool]
$ python3 visitor/gentufa_visitor.py "mi klama le zarci" | jq

{
  "sentence": "mi klama le zarci",
  "segments": [
    {
      "mi": {
        "definition": "I/me, we/us",
        "type": "cmavo / KOhA selma'o (mi series) / pronoun"
      }
    },
    {
      "klama": {
        "definition": "$x_{1}$ comes/goes to destination $x_{2}$ from origin $x_{3}$ via route $x_{4}$ using means/vehicle $x_{5}$.",
        "glosswords": [ {"word": "come", "sense": null} ],
        "type": "gismu",
        "sumti": [ "$x_{1}$", "$x_{2}$", "$x_{3}$", "$x_{4}$", "$x_{5}$" ]
      }
    },
    {
      "le zarci": {
        "type": "description brivla",
        "segments": [
          {
            "le": {
              "definition": "description marker / 'the'",
              "type": "cmavo / LE selma'o"
            }
          },
          {
            "zarci": {
              "definition": "$x_{1}$ is a market/store/exchange/shop(s) selling/trading (for) $x_{2}$, operated by/with participants $x_{3}$.",
              "glosswords": [
                { "word": "market", "sense": null },
                { "word": "shop", "sense": "store" }
              ],
              "type": "gismu",
              "sumti": [ "$x_{1}$", "$x_{2}$", "$x_{3}$" ]
            }
          }
        ]
      }
    }
  ]
}
\end{lstlisting}

\newpage

\begin{lstlisting}[caption=GentufaVisitor class being used by a Python script]
>>> from grammar_parser import Gentufa
>>> from visitor import GentufaVisitor
>>> parse_tree = Gentufa().get_parsed_sentence("mi klama le zarci")
>>> visitor = GentufaVisitor(parse_tree)
>>> visitor.get_json_output()
'{"sentence": "mi klama le zarci", "segments": [{"mi": {"definition": "I/me, we/us", "type": "cmavo / KOhA selma\'o (mi series) / pronoun"}}, {"klama": {"definition": "$x_{1}$ comes/goes to destination $x_{2}$ from origin $x_{3}$ via route $x_{4}$ using means/vehicle $x_{5}$.", "glosswords": [{"word": "come", "sense": null}], "type": "gismu", "sumti": ["$x_{1}$", "$x_{2}$", "$x_{3}$", "$x_{4}$", "$x_{5}$"]}}, {"le zarci": {"type": "description brivla", "segments": [{"le": {"definition": "description marker / \'the\'", "type": "cmavo / LE selma\'o"}}, {"zarci": {"definition": "$x_{1}$ is a market/store/exchange/shop(s) selling/trading (for) $x_{2}$, operated by/with participants $x_{3}$.", "glosswords": [{"word": "market", "sense": null}, {"word": "shop", "sense": "store"}], "type": "gismu", "sumti": ["$x_{1}$", "$x_{2}$", "$x_{3}$"]}}]}}]}'
\end{lstlisting}