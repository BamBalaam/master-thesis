\chapter{Results and closing remarks}

\vspace{0.5cm}

The conjunction of the parser, the visitor, and the PEG created in this thesis currently cover most of the language described in
\citetitle{cowan1997complete} until chapter 7 (with a few excursions into other chapters). 21 chapters exist in the book,
but only 17 chapters that address grammar, which means that this thesis achieved to create tools which help students of
Lojban to understand more than 40\% of the major reference book on the language. \newline

The final test coverage is as follows:

\begin{itemize}
    \item 200 sentences from 7 chapters were selected;
    \item of these 200 sentences, 186 were parsed and visited correctly;
    \item thus the apparent success rate of the parser, for this dataset is 93\%
\end{itemize}

However, it is important to note that all sentences from Chapter 2 of the reference book are validated, and that chapter
gives a very broad tour of the language, so the tools developed allow true beginner students to study in depth the most
important chapter of all.\newline

Looking back at the aims and objectives set in Chapter \ref{chap:aims-and-objectives} of this thesis, one must ponder if
they were met. Let us revisit them and evaluate:

\begin{itemize}
\item \textbf{State of the Art (Part \ref{part:state-of-the-art})}: A broad overview of the study of constructed languages, and Lojban
in specific, was created in order to gain deeper understanding on the subject.
\item \textbf{Creation of a Python-Lojban toolkit (Part \ref{part:python-toolkit})}: Three different tools were created, each
serving a separate purpose but working in collaboration. These tools allow students, while learning Lojban, to gather word definitions,
test if the sentences they are writing are correct, and decompose complex sentences they might struggle to understand.
\item \textbf{Creation of a simplified Lojban PEG (Part \ref{part:creating-a-peg})}: The created grammar is, of course, not complete.
Several grammar aspects are still missing, it over-simplifies some of the rules, and possibly validates sentences that other parsers wouldn't.
However, while creating it, I was able to understand Lojban much better than with the few tutorials existing online, which means that this way
of studying (reference book + companion grammar) does have at least some anecdotal success. The only way to assess the veracity of this point
would be to convince people to study Lojban and do a parallel study.
\end{itemize}

The overarching objective was to create a learning companion for students. It is fair to say that at least this humble objective was met,
and that these tools are a good starting point for continued work in the subject.\newline

Learning a language is never easy, and for one as complex as Lojban this expression is an understatement. As someone who has always been interested
in (self-)education, I hope that the work accomplished in this thesis, and the future work planned ahead, might help prospective
learners of Lojban one day.