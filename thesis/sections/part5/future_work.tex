\chapter{Results and Future work}

\vspace{0.5cm}

\section{Results}

The conjunction of the parser, the visitor, and the PEG created in this thesis currently cover most of the language described in
\citetitle{cowan1997complete} until chapter 7 (with a few excursions into other chapters). 21 chapters exist in the book,
but only 17 chapters that address grammar, which means that this thesis achieved to create tools which help students of
Lojban to understand more than 40\% of the major reference book on the language.\newline

However, it is important to note that Chapter 2 of the reference book gives a very broad tour of the language, so the tools
developed allow true beginner students to study in depth the most important chapter of all.\newline

The final test coverage is the following: X sentences from Y chapters parsed and visited correctly.

\newpage

\section{Future Work}

Creating tools for a constructed language is a pharaonic project, and a lot more could be done for this project,
but this thesis could not cover more due to lack of time. I would enjoy continuing to work on these tools and grammar,
mainly on the following aspects:

\begin{itemize}
\item \textbf{Address more aspects of the grammar}: go further along the companion book and parse even more sentences;
\item \textbf{Visualisation module}: the output generated by the "visitor" module could be presented in a visual interface,
to be even more pedagogic for prospective students of the language;
\item \textbf{Improve visitor functions}: the current setup is quite time-consuming, some investigation is needed to see if
it is possible to extend the Parsimonious\footcite{parsimonious} python library in order to create functions which are more generic;
\item \textbf{Get definitions of cmavo/selma'o from the grammar itself}: another frustration encountered during the development of
the visitor module was the need to duplicate definitions from the grammar into the code - another possible extension and
open-source contribution to the Parsimonious library could be modify how the grammar rules are gathered and allow to access
comments from within the Parse Tree nodes.
\end{itemize}