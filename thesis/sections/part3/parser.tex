\chapter{Parsing sentences: the ``grammar\_parser'' module}
\label{chap:parser}

\section{Methodology}

As seen in Chapter \ref{chap:existing-resources-and-tools}, existing parsers online are not very user-friendly. They are either
over-engineered or written in very outdated technical source-codes. Additionally, the Parse Tree they output is very hard to read,
even for simple sentences. Our main objective with this module is to create a parser that is both fast and outputs a fairly readable parse tree,
in order for students to be able to test if their simple sentences are correct or not.

\section{Breakdown of the module}
\label{sec:parsing_lojban_sentences}

The code produced is found at Annex \ref{appendix:gentufa-annex}. You may read it along the following explanations.

\subsection*{README.md file}

Allows users to understand what the module does and how it is used.

\subsection*{\_\_init\_\_.py file}

Allows other modules to import the Gentufa class without calling the file it is contained in explicitly.

\subsection*{gerna.peg file}

Contains the simplified Parsing Expression Grammar (PEG) created for this project.
We shall ignore this file for now, as it will be discussed in a lot more detail in Part \ref{part:creating-a-peg}.

\subsection*{gentufa.py file}

This file defines the \textbf{Gentufa} class which, once instantiated, allows to parse Lojban sentences through an helper function.
The parse tree returned is, thanks to Parsimonious \footcite{parsimonious} and the simplified PEG presented later in this thesis,
very readable.\newline

The following list is a brief description of all the functions in the class:

\begin{itemize}
\item \textbf{\_\_init\_\_ :} instantiates a \textbf{Gentufa} class object by doing four simple tasks:
    \begin{itemize}
        \item instantiate a \textbf{Jbovlaste} instance to have a dictionary on hand
        \item open the grammar file and store its text
        \item augment the grammar text with dictionary words, by finding specific tags in it
        \item feed that grammar text to the Grammar class from Parsimonious, and keep an instance of it to parse sentences
    \end{itemize}
\item \textbf{get\_parsed\_text:} receives some text as an argument and uses the Grammar object stored in the class, instantiated in the
\_\_init\_\_ function, to parse the text and return a parse tree
\end{itemize}

Outside of the class, we can find an "if-clause".
This is the entrypoint of the file if it is executed directly through a terminal.
In that case, the Gentufa class is instantiated and the "get\_parsed\_text" function
is executed, with whatever sentence was given as an argument in the call. It returns an explicit
error message if the sentence does not generate a valid parse tree.

\newpage

\section{Examples of usage}

This module can be used in two different ways: as a command-line interface (CLI) tool
from a terminal, or as a class for short scripts or other modules to use.

\begin{lstlisting}[caption=Gentufa module being used as a command-line interface tool]
$ python3 grammar_parser/gentufa.py "mi tavla do"

<Node called "LOJBAN_TEXT" matching "mi tavla do EOL">
    <Node called "SENTENCE" matching "mi tavla do ">
        <Node called "LOJBAN_WORDS_OR_EXPRESSIONS" matching "mi ">
            <Node called "CMAVO" matching "mi ">
                <Node called "SELMAhO_KOhA" matching "mi ">
                    <Node called "KOhA_MI_SERIES" matching "mi ">
                        <Node called "MI" matching "mi ">
                            <Node matching "mi">
                            <RegexNode called "_" matching " ">
        <Node called "LOJBAN_WORDS_OR_EXPRESSIONS" matching "tavla ">
            <Node called "TANRU_OR_BRIVLA" matching "tavla ">
                <Node called "BRIVLA" matching "tavla ">
                    <Node called "GISMU" matching "tavla ">
                        <Node matching "tavla ">
                            <Node matching "tavla">
                            <RegexNode called "_" matching " ">
        <Node called "LOJBAN_WORDS_OR_EXPRESSIONS" matching "do ">
            <Node called "CMAVO" matching "do ">
                <Node called "SELMAhO_KOhA" matching "do ">
                    <Node called "KOhA_MI_SERIES" matching "do ">
                        <Node called "DO" matching "do ">
                            <Node matching "do">
                            <RegexNode called "_" matching " ">
    <Node matching "">
    <Node called "EOL" matching "EOL">
\end{lstlisting}

\begin{lstlisting}[caption=Gentufa class being used by a Python script]
>>> from grammar_parser import Gentufa
>>> parser = Gentufa()
>>> parsed_text = parser.get_parsed_text("mi tavla do")
>>> type(parsed_text)
<class 'parsimonious.nodes.Node'>
>>> print(parsed_text)
<Node called "SENTENCE" matching "mi tavla do EOL">
    (...) ---> Same representation as the previous example
    <Node called "EOL" matching "EOL">
\end{lstlisting}