\chapter{Tagging and labeling parsed sentences: the ``visitor'' module}
\label{chap:visitor-module}

\section{Methodology}

Parsimonious \footcite{parsimonious} has a very interesting feature and API called "visitor". This framework
allows to explore a Parsimonious parse tree, like the one generated in the previous module, and represent nodes in
whichever way we want.\newline

Our approach in this module will be to explore the parse tree and identify important nodes, in order to represent
their function in the grammar and their definition. All these nodes will be represented as elements in an overarching
JSON object. The final objective is for the JSON object to be human-readable, in order for students to identify the
important elements of a Lojban sentence and what they mean (both in definition and in the grammar).\newline

What we achieve is a mix of Parts-Of-Speech tagging (giving each word a grammatical category) and Semantic role labeling,
because tagging verbs allows us, thanks to their definition, to know which word accomplishes what role in the sentence.

\section{Breakdown of the module}
\label{sec:visitor-code-breakdown}

The code produced is found at Annex \ref{appendix:gentufa-visitor-annex}. You may read it along the following explanations.

\subsection*{README.md file}

Allows users to understand what the module does and how it is used.

\subsection*{\_\_init\_\_.py file}

Allows other modules to import the GentufaVisitor class without calling the file it is contained in explicitly.

\subsection*{gentufa\_visitor.py file}

This file defines the \textbf{GentufaVisitor} class which, once instantiated with a parse tree as an argument, allows to
explore it and returns a human-readable JSON format of the sentence with all nodes annotated. This class is a child of the
NodeVisitor class from Parsimonious.\newline

The following list is a brief description of all the functions in the class:

\begin{itemize}
\item \textbf{\_\_init\_\_ :} instantiates a \textbf{GentufaVisitor} class object by doing a few simple tasks
  \begin{itemize}
    \item instantiate a \textbf{Jbovlaste} instance to have a dictionary on hand
    \item creates the base format of the output, with the full text and an empty list of sentences
    \item runs the parent class function called "visit", giving it the parse tree as an argument, which will execute
    all the visitor functions we'll dynamically load from other files (see below).
  \end{itemize}
\item \textbf{get\_output:} returns the final output after the "visit" function has run
\item \textbf{get\_json\_output:} same as above, but formatted as a JSON object
\item \textbf{generic\_visit:} required by the parent class
\end{itemize}

Outside of the class, we can find an "if-clause" and a decorator function.\newline

The if-clause is the entrypoint of the file if it is executed directly through a terminal.
In that case, the sentence passed as an argument is given to a Gentufa class instance, in order to
retrieve a parse tree, and the get\_json\_output function is executed.\newline

The decorator allows to dynamically add a list of functions to a class, in order to avoid overburdening
its definition and separate class functions into standalone files (instead of having one gigantic class definition).

\subsection*{visitor\_functions folder}

This folder contains all the visitor functions used by GentufaVisitor in order to traverse the parse tree. The files included
in the folder are the following:

\begin{itemize}
  \item \textbf{\_\_init\_\_.py :} aggregates all functions from the files below in a list, in order for the decorator described above
  to add them to the \textbf{GentufaVisitor} class dynamically.
  \item \textbf{base\_grammar.py:} visitor functions for the basic rules in our grammar
  \item \textbf{selmaho.py:} visitor functions for the selma'o rules in our grammar (see Part \ref{part:creating-a-peg}
  to understand what selma'o are)
\end{itemize}

\newpage

\section{Examples of usage}

This module can be used in two different ways: as a command-line interface (CLI) tool
from a terminal, or as a class for short scripts or other modules to use.

\begin{lstlisting}[caption=GentufaVisitor module being used as a command-line interface tool]
$ python3 visitor/gentufa_visitor.py "mi tavla do" | jq

{
  "text": "mi tavla do",
  "sentences": [
    {
      "sentence": "mi tavla do ",
      "segments": [
        {
          "mi": {
            "definition": "I/me",
            "type": "cmavo / selma'o KOhA (mi series) / personal pronouns"
          }
        },
        {
          "tavla": {
            "definition": "$x_{1}$ talks/speaks to $x_{2}$ about subject $x_{3}$ in language $x_{4}$.",
            "glosswords": [
              {
                "word": "talk", "sense": "any speech"
              }
            ],
            "type": "gismu",
            "sumti": [
              "$x_{1}$", "$x_{2}$", "$x_{3}$", "$x_{4}$"
            ]
          }
        },
        {
          "do": {
            "definition": "You",
            "type": "cmavo / selma'o KOhA (mi series) / personal pronouns"
          }
        }
      ]
    }
  ]
}
\end{lstlisting}

\newpage

\begin{lstlisting}[caption=GentufaVisitor class being used by a Python script]
>>> from grammar_parser import Gentufa
>>> from visitor import GentufaVisitor
>>> parse_tree = Gentufa().get_parsed_text("mi tavla do")
>>> visitor = GentufaVisitor(parse_tree)
>>> visitor.get_json_output()
'{"text": "mi tavla do", "sentences": [{"sentence": "mi tavla do ", "segments": [{"mi": {"definition": "I/me", "type": "cmavo / selma\'o KOhA (mi series) / personal pronouns"}}, {"tavla": {"definition": "$x_{1}$ talks/speaks to $x_{2}$ about subject $x_{3}$ in language $x_{4}$.", "glosswords": [{"word": "talk", "sense": "any speech"}], "type": "gismu", "sumti": ["$x_{1}$", "$x_{2}$", "$x_{3}$", "$x_{4}$"]}}, {"do": {"definition": "You", "type": "cmavo / selma\'o KOhA (mi series) / personal pronouns"}}]}]}'
\end{lstlisting}