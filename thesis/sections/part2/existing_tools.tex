\chapter{Existing resources and tools for Lojban}
\label{chap:existing-resources-and-tools}

\section{Grammars and Parsers}
\label{section:lojban-grammars}

As mentioned in \ref{subsec:learning-lojban}, Lojban's grammar has gone through several iterations since its initial creation
due to dissenting opinions about the state of the language and its future.
A big list of tools and websites exist to parse and represent Lojban sentences with non-official and experimental grammars,
none of which take a pedagogical standpoint as the simplified one which is created in this thesis, but the following are the most famous ones:

\subsection*{Camxes}

Camxes \footcite{camxes} is one of the first unofficial grammars for Lojban. Created in reaction to his opinionated stance about the official
Lojban Formal Grammar, \citeauthor{camxes}'s stance is that the formal grammar has many flaws, context free grammars are not the right formalism
for Lojban, and a few aspects of the grammar need to be extended by external scripts in order to work properly.

\subsection*{CamxesJS}

Camxes.js \footcite{camxesjs} supersedes Camxes as a rewrite in a more modern language than Java, with a few tweaks to the grammar. This version
will influence many others, such as the next one in the list.

\subsection*{zantufa}

Zantufa \footcite{zantufa} is one of the latest parsers that exists, which includes a lot of experimental and unofficial grammar aspects.

\section{Dictionary}
\label{sec:dictionary}

Many Lojban dictionaries exist, almost all of them online though (not many tools like the one we'll create
in this thesis for browserless use).\newline

The three most famous ones are the following:

\begin{itemize}
\item "la jbovlaste" \footcite{jbovlaste}: the de-facto standard Lojban dictionary, collaborative and multilingual;
\item "la sutysisku" \footcite{sutysisku}: multilingual interface for "la jbovlaste" database with some algorithms improving search relevance;
\item "la vlasisku" \footcite{vlasisku}: english only, but the interface is way more user-friendly and modern.
\end{itemize}