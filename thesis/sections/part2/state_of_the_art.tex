\chapter{Existing research and references}

\newpage \thispagestyle{empty} \ \newpage

\chapter{Existing resources and tools for Lojban}

\vspace{0.5cm}

\section{Grammars}
\label{section:lojban-grammars}

Camxes

https://mw.lojban.org/papri/camxes

Java

Camxes.js

http://masatohagiwara.net/camxes.js/

javascript

La ilmentufa

https://mw.lojban.org/papri/la\_ilmentufa


\section{Parsers}

Official LLG Parser

https://mw.lojban.org/papri/Official\_LLG\_Parser

C78 source code
or
MS-DOS executable

\begin{itemize}
    \item BPFK Parser: official parser for Lojban maintained by the Logical Language Group (LLG) - implements the official grammar and is intended to be the reference parser for the language.
    \item jlöi parser
    \item plise parser
    \item Lojban Natural Language Toolkit (LONU)
    \item Zantufa
    \item lojban-parsing
\end{itemize}

\section{Dictionary}
\label{sec:dictionary}

jbovlaste is the Lojban dictionary. \\

la sutysisku, interface for la jbovlaste database with some algorithms improving search relevance. \\

Lojban Online Dictionary Query. The jbovlaste search engine.\\