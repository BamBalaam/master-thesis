\chapter{State of the art}

\vspace{0.5cm}

\section{Existing research}

\subsection{Machine Translation through constructed languages}

The Study of Machine Translation Aspects Through Constructed Languages

describes a software system that performs bidirectional machine translation between two constructed languages, which can be used to develop easy and almost natural communication interfaces with robots. The authors also discuss the challenges of machine translation of natural languages and how the usage of constructed languages can be an easier way to train linguistic engineers in developing machine translation software. Additionally, the paper provides examples of constructed languages and discusses the importance of choosing a suitable pair of languages for a project.

Lojban as a Machine Translation Interlanguage in the Pacific

discusses the use of Lojban as an interlingua in machine translation in the Pacific region. It explains the difference between transfer and interlingua strategies in machine translation and argues that Lojban, as a constructed language with a simple grammar and unambiguous semantics, can serve as an effective interlingua. The paper also discusses the limitations of using Lojban as an interlingua and suggests possible solutions to overcome these limitations. Overall, the paper provides a comprehensive overview of the benefits and challenges of using Lojban as an interlingua in machine translation.

\subsection{Language processing of Lojban}

Meeting the Computer Halfway: Language Processing in the Artifical Language Lojban

the use of the artificial language Lojban in language processing and its advantages in facilitating communication between humans and computers. It highlights the challenges faced by natural language researchers in dealing with ambiguity, polysemy, and vague grammar rules. Lojban's grammar is unambiguous, allowing for easy parsing using tools like YACC. The system described in the PDF, called JIMPE, is a Python program composed of various modules that pass information between each other. The ultimate goal of the system is to enhance communication between humans and computers.

\subsection{Other Lojban-related sources}

Semantic parsing using Lojban – On the middle ground between semantic ontology and language

\section{Existing resources and tools}

\subsection{Parsers}

Official LLG Parser

https://mw.lojban.org/papri/Official\_LLG\_Parser

C78 source code
or
MS-DOS executable


Camxes

https://mw.lojban.org/papri/camxes

Java

Camxes.js

http://masatohagiwara.net/camxes.js/

javascript

La ilmentufa

https://mw.lojban.org/papri/la\_ilmentufa

