\chapter{Aims and objectives of this thesis}
\label{chap:aims-and-objectives}

\vspace{0.5cm}

At its root, computer science had a singular aim --- automating calculations for a broad spectrum of use-cases.
However with time and with the emergence of better, faster, and stronger computers, researchers understood that computers could be used to achieve
far more complex tasks. Several diverse research fields made their appearance, and among these some realised that computing could help us structure the
flow of information, in addition to making calculations based on it. \newline

The explosion of tasks requiring the processing of large amounts of structured data inside of companies, and much later the arrival of the Internet,
forced us to debate and research subjects such as: how to gather information, how to manage and store it, analyse and interpret it, explore it, etc...
A sizeable chunk of this information comes to us in the form of text and language, in their many forms. To answer these needs, language engineering became a
central research field in a new society, one that had migrated from a purely industrial society to an ``information society''.\newline

This thesis aims to accompany the reader through the exploration of the intersection between two fields: linguistics (constructed languages)
and computer science (language processing, parsing expression grammars, and others). Lojban, being a language purely based in logic, seems like a very
interesting candidate to help us through this exploration. Unfortunately, as seen in \ref{subsec:learning-lojban}, Lojban is not an easy language to study.
What if learners had a set of tools which helped them understand Lojban's grammar, through analysis and visualisation of parsed sentences?
This thesis aims to create such a set of tools.\newline

\newpage

To do so, some objectives are needed. They are the following:
\begin{itemize}
\item Outline the state of the art and existing tools/resources for Lojban (Part \ref{part:state-of-the-art})
\item Create tools, written in the Python programming language, to explore Lojban sentences and help learners of this
language to understand how it is structured (Part \ref{part:python-toolkit})
\item Create a simplified Parsing Expression Grammar (PEG) for Lojban, used by the tools created,
which allows the reader to learn the basics of Lojban's grammar while reading a set of parsing expressions (Part \ref{part:creating-a-peg})
\item Examine if these objectives were met, and outline future work (Part \ref{part:conclusion})
\end{itemize}