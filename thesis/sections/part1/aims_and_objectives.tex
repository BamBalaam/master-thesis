\chapter{Aims and objectives of this thesis}

At its root, computer science had a singular aim --- automating calculations for a broad spectrum of use-cases.
However with time and with the emergence of better, faster, and stronger computers, researchers understood that computers could be used to achieve
far more complex tasks. Several diverse research fields made their appearance, and among these some realised that computing could help us structure the
flow of information, in addition to making calculations based on it. \newline

The explosion of tasks requiring the processing of large amounts of structured data inside of companies, and much later the arrival of the Internet,
forced us to debate and research subjects such as: how to gather information, how to manage and store it, analyse and interpret it, explore it, etc...
A sizeable chunk of this information comes to us in the form of text and language, in their many forms. To answer these needs, language engineering became a
central research field in a new society, one that had migrated from a purely industrial society to an ``information society''.\newline

The main aim of this thesis is to explore the intersection between the two fields presented in this introduction: linguistics, around the subject of a
particular constructed language, and computer science, which allows language processing. The first objective of this thesis is to outline the state of the
art and existing parsers for Lojban, list both their flaws and positive aspects, and define what kind of new parser would be interesting to develop.
Using the Python programming language, a set of tools will be created in order to parse and visualise the structure of given sentences, in order to analyse them.\newline