\chapter{Introduction}

\vspace{0.5cm}

As seen in Chapter \ref{chap:existing-resources-and-tools}, the existing formal grammars for Lojban are extremely complete.
They cover the entirety of the language, both in its morphology and syntax/semantics, which means they can even validate words that don't exist yet
as long as they fit the language's rules (as mentioned in \ref{subsec:brief-intro-to-lojban}, an algorithm for "word creation" exists).
However, this completion makes these formal grammars extremely complex to read for a beginner, and are simply tools to be used by machines.
As an example, here is an extract of the first few "root nodes" of 'camxes', a famous Lojban PEG (see Chapter \ref{chap:existing-resources-and-tools}
for more details about it):

\begin{lstlisting}[caption=Extract of 'camxes' - a famous Lojban PEG]
paragraphs <- paragraph (NIhO-clause+ free* su-clause* paragraphs)?
paragraph <- (statement / fragment) (I-clause !jek !joik !joik-jek free* (statement / fragment)?)*
statement <- statement-1 / prenex statement
statement-1 <- statement-2 (I-clause joik-jek statement-2?)*
statement-2 <- statement-3 (I-clause (jek / joik)? stag? BO-clause free* statement-2)? / statement-3 (I-clause (jek / joik)? stag? BO-clause free*)?
statement-3 <- sentence / tag? TUhE-clause free* text-1 TUhU-clause? free*
fragment <- prenex / terms VAU-clause? free* / ek free* / gihek free* / quantifier / NA-clause !JA-clause free* / relative-clauses / links / linkargs
prenex <- terms ZOhU-clause free*
\end{lstlisting}

\newpage

Even advanced users of Lojban might struggle to understand all the intricacies at play in just the first few lines, never mind a beginner.\newline

What if there existed a formal grammar which people could read along, while learning Lojban, in order to gather a deeper
understanding of the language? For any other language, using a formal grammar file as a learning companion would be a very obscure method.
However, Lojban being a language that can be parsed just as a programming language would, it would be very interesting to build a formal grammar
which would help a prospective student understand the intrinsic rules behind the language's grammar. \newline

Of course, improving the reading comprehension of a formal grammar reduces the complete aspect of it:

\begin{itemize}
  \item It is almost impossible to address all rules of the grammar, as some of them are extremely complex;
  \item Ignoring the morphology of words means that unknown but valid words won't be accepted;
\end{itemize}

However, as a beginner's tool, such a grammar could really improve the steep learning curve of Lojban, in my opinion.\newline

The next chapter will explore the main reference for Lojban \footcite{cowan1997complete}, in its 2016 revised version,
and write a simplified parsing expression grammar (PEG) for a subset of the language.