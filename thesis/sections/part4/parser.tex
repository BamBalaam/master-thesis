\chapter{Parsing sentences: the ``grammar\_parser'' module}
\label{chap:parser}

\section{Methodology}

% https://en.wikipedia.org/wiki/Parse_tree

\newpage

\section{Examples of usage}

This module can be used in two different ways: as a class for short scripts or other modules to use, or as a command-line interface (CLI) tool.

\begin{lstlisting}[caption=Gentufa class being used by a Python script]
>>> from grammar_parser import Gentufa
>>> parser = Gentufa()
>>> parsed_sentence = parser.get_parsed_sentence("mi klama le zarci")
>>> type(parsed_sentence)
<class 'parsimonious.nodes.Node'>
>>> print(parsed_sentence)
<Node called "SENTENCE" matching "mi klama le zarci EOL">
    <Node matching "mi klama le zarci ">
        <Node called "LOJBAN_WORDS_OR_EXPRESSIONS" matching "mi ">
            <Node called "CMAVO_WITH_OR_WITHOUT_MODIFIERS" matching "mi ">
                <Node called "CMAVO" matching "mi ">
                    <Node called "KOhA_MI_SERIES" matching "mi ">
                        <Node called "MI" matching "mi ">
                            <Node matching "mi">
                            <RegexNode called "_" matching " ">
        <Node called "LOJBAN_WORDS_OR_EXPRESSIONS" matching "klama ">
            <Node called "BRIVLA_WITH_OR_WITHOUT_MODIFIERS" matching "klama ">
                <Node called "BRIVLA" matching "klama ">
                    <Node called "GISMU" matching "klama ">
                        <Node matching "klama ">
                            <Node matching "klama">
                            <RegexNode called "_" matching " ">
        <Node called "LOJBAN_WORDS_OR_EXPRESSIONS" matching "le zarci ">
            <Node called "BRIVLA_WITH_OR_WITHOUT_MODIFIERS" matching "le zarci ">
                <Node called "DESCRIPTION_BRIVLA" matching "le zarci ">
                    <Node called "SELMAhO_LE" matching "le ">
                        <Node matching "le">
                        <RegexNode called "_" matching " ">
                    <Node called "BRIVLA" matching "zarci ">
                        <Node called "GISMU" matching "zarci ">
                            <Node matching "zarci ">
                                <Node matching "zarci">
                                <RegexNode called "_" matching " ">
                    <Node matching "">
    <Node called "EOL" matching "EOL">
\end{lstlisting}

\newpage

\begin{lstlisting}[caption=Gentufa module being used as a command-line interface tool]
$ python3 grammar_parser/gentufa.py "mi klama le zarci"

<Node called "SENTENCE" matching "mi klama le zarci EOL">
    <Node matching "mi klama le zarci ">
        (...) ---> Same representation as previous example
    <Node called "EOL" matching "EOL">
\end{lstlisting}

\section{Breakdown of the module}
\label{sec:parsing_lojban_sentences}

The code produced is found at Annex \ref{appendix:gentufa-annex}.

TODO