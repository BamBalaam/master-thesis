\chapter{Learning Lojban by writing a PEG}
\label{chap:writing-a-peg}

\section{Methodology}

One of the aims of this thesis (see Chapter \ref{chap:aims-and-objectives}) is to write a Grammar for Lojban sentences which is
a very simple and good learning companion for Lojban students.
Thankfully, the Parsimonious \footcite{parsimonious} library uses a simplified format of Extended Backus-Naur form to write
Parsing Expression Grammars. Its syntax is indeed simple:

\begin{itemize}
    \item rules have the form rule = literal or conjunction of expressions
    \item literals are delimited with quotes (e.g. "word")
    \item regular expressions are valid literals
    \item alternative expressions are possible, by separating them with "/", and follow the "first match first consume" rule
    \item "?" delimits that an expression is optional
    \item "*" delimits that zero or more expressions might match
    \item "+" delimits that one or more expressions might match
    \item "\{n,m\}" delimits that n to m repetitions might match
    \item parenthesis are used for grouping
\end{itemize}

With these rules in mind, let's explore the creation of a simplified PEG for Lojban.

\newpage

\section{Breakdown of the PEG}

In the following pages, we'll explore the grammar rules written without showing all of the pedagogical contextual comments
surrounding them. As they are an integral part of the process, we invite the reader to take a look at Annex \ref{appendix:peg-annex}
for the whole listing of the grammar.

\subsection{Base rules and morphology}

Two small helpers which will be important to avoid parsing issues are added:

\begin{lstlisting}[caption=PEG helpers]
_ = ~r"\s+"
EOL = "EOL"
\end{lstlisting}

The root node - a Lojban text is composed of one or more sentences, separated by the characters ".i" (known as selma'o I), concluded by an End Of Line marker:

\begin{lstlisting}[caption=PEG rules for Lojban text]
LOJBAN_TEXT = SENTENCE SUBSEQUENT_SENTENCE* EOL
SUBSEQUENT_SENTENCE = SELMAhO_I SENTENCE
SELMAhO_I = ".i" _
\end{lstlisting}

A sentence contains one or more Lojban words or expressions:

\begin{lstlisting}[caption=PEG rules for Lojban sentences]
SENTENCE = (LOJBAN_WORDS_OR_EXPRESSIONS)+
\end{lstlisting}

There are 3 basic parts of speech - cmavo, brivla (tanru these are chained) and cmene:

\begin{lstlisting}[caption=PEG rules for basic parts of speech]
LOJBAN_WORDS_OR_EXPRESSIONS = CMAVO / TANRU_OR_BRIVLA / CMENE
TANRU_OR_BRIVLA = TANRU / BRIVLA
TANRU = BRIVLA{2,}
\end{lstlisting}

Cmavo are the structure words that hold the language together. These are grouped in families called selma'o and
will be presented in more detail in the next section:

\begin{lstlisting}[caption=CMAVO rule]
CMAVO = (
    SELMAhO_ZEI /
    SELMAhO_BO / SELMAhO_KE / SELMAhO_KEhE / SELMAhO_SE /
    SELMAhO_LE / SELMAhO_LA / SELMAhO_KU / SELMAhO_LAhE / SELMAhO_LUhU / SELMAhO_COI /
    SELMAhO_KOhA / SELMAhO_GOhA /
    SELMAhO_CU / SELMAhO_FA /
    SELMAhO_VA / SELMAhO_FAhA / SELMAhO_PU / SELMAhO_ZI /
    SELMAhO_UI /
    SELMAhO_BY /
    SELMAhO_BAhE
)
\end{lstlisting}

Brivla are predicate words. They are the equivalent of nouns, verbs, adjectives and adverbs and are classified in three families:
gismu (core words), lujvo (compound words) and fu'ivla (borrowed verbs or loan words).\newline

These are the words which are augmented in the grammar during the parsing process via the dictionary we created
(see Chapter \ref{chap:parser}). The words surrounded by curly braces are replaced by a long list of literals, separated
by an alternative delimiter ("/").

\begin{lstlisting}[caption=BRIVLA rule]
BRIVLA = GISMU / LUJVO / FUhIVLA
GISMU = {{gismu}}
LUJVO = {{lujvo}}
FUhIVLA = {{fu'ivla}}
\end{lstlisting}

\newpage

Cmene are words used as a name (for things or people). For these, we'll use a Regular Expression in order to follow Lojban's rules:

\begin{lstlisting}[caption=CMENE rule]
CMENE = (~r"\.?[A-Za-z]+\.?") _
\end{lstlisting}

\subsection{The many cmavo and selma'o}

Listing all of the cmavo and selma'o added into the grammar would be verbose and annoying for this thesis' readers. However, they are
one of the main selling points of this grammar since they introduce commentary that helps students understand them. Thus, a few examples will
be presented in order for the reader to get an idea of how these rules are presented:

\begin{lstlisting}[caption=SELMAhO\_ZEI rule]
# selma'o ZEI (Section 4.6): join two words to create a lujvo (e.g. "xy. zei kantu" = X-ray)
ZEI = "zei" _
SELMAhO_ZEI = (GISMU / FUhIVLA / CMENE) ZEI (GISMU / FUhIVLA / CMENE)
\end{lstlisting}

\begin{lstlisting}[caption=KOhA\_MI\_SERIES rule]
# selma'o KOhA: mi-series (Section 7.2): Personal pronouns
KOhA_MI_SERIES = (MI / MIhO / MIhA / MAhA / DO / DOhO / KO)
MI = "mi" _     # I/me
MIhO = "mi'o" _ # You and I
MIhA = "mi'a" _ # I and others, we but not you
MAhA = "ma'a" _ # You and I and others
DO = "do" _     # You
DOhO = "do'o" _ # You and others
KO = "ko" _     # You - imperative
\end{lstlisting}