\chapter{Defining words: the ``dictionary'' module}
\label{chap:creating_a_dictionary}

\section{Methodology}

Described in Chapter \ref{chap:existing-resources-and-tools}, La Jbovlaste is the official Lojban dictionary published by the LLG and available online \footcite{jbovlaste}.
This dictionary allows to export the entirety of its data in a XML format at the "XML Export" page \footcite{jbovlaste-export}.
The format of the XML file is as follows (showing the example of a single word, jbovlaste itself):

\begin{lstlisting}[caption=Example of word as defined in the official Lojban dictionary XML export]
<?xml version="1.0" encoding="UTF-8"?>
<?xml-stylesheet type="text/xsl" href="jbovlaste.xsl"?>
<dictionary>
  (...)
  <valsi word="jbovlaste" type="lujvo">
    (...)
    <definition>$x_1=li_1$ is a list of words $x_2=v_1$ in Lojban ($v_3$=lo lojbo), in order $x_3=li_3$, preserved in medium $x_4=li_4$.</definition>
    <definitionid>32240</definitionid>
    <notes>In a la-description, jbovlaste refers to the online dictionary editing system.</notes>
    <glossword word="dictionary" sense="lojban word list" />
  </valsi>
  (...)
</dictionary>
\end{lstlisting}

In this first module in the Python-Lojban toolkit, a Python class is created in order to parse this XML file and aggregate all words present,
their type, and their definition. Once that dictionary is instantiated, functions of this class will allow to get word definitions
(in a raw or "prettified" format for displaying on a terminal) or Python objects used in other modules.

\section{Breakdown of the code in the module}

The code produced is found at Annex \ref{appendix:jbovlaste-annex}.

\subsection*{README.md file}

Allows users to understand what the module does and how it is used.
Also includes a citation to where the raw XML file used comes from.

\subsection*{\_\_init\_\_.py file}

Allows other modules to import the Jbovlaste class without calling the file it is contained in explicitly.

\subsection*{jbovlaste.py file}

This file defines the Jbovlaste class which, once instantiated, allows to fetch
words in Lojban through helper functions. The structure returned contains the type of word it is,
a definition, and optional glossary words in the language of the dictionary.
As of right now, only English is available as a target language but making other
languages available is fairly trivial (although other languages are not as complete
in the official "La Jbovlaste").\newline

\newpage

The following list is a brief description of all the
functions in the class:

\begin{itemize}
\item \textbf{\_\_init\_\_}: instantiates the class object by building a dictionary of
Lojban words stored in a Python "dict" data type:
  \begin{itemize}
    \item opens the XML file and finds all nodes named "valsi" ("word" in Lojban);
    \item constructs the dictionary by classifying words by their type, and adds
    each word into a sub-dictionary of that word type;
    \item extracts from each word its definition and some "glosswords".
  \end{itemize}
\item \textbf{get\_whole\_dict}: returns the entire Python "dict" data type that stores
the words.
\item \textbf{get\_word\_struct}: tries to find a word, given as a parameter of the function,
in the dictionary; if found, returns an object with all known information about it;
if not found, returns a KeyError (one of Python's default errors).
\item \textbf{get\_word\_pretty}: gathers the object returned by get\_word\_struct and returns
a formatted string with the information contained within to be displayed.
\item \textbf{decompose\_definition} and \textbf{get\_definition\_arguments}: uses a Regular Expression
in order to extract and list arguments ("sumti") from a definition
\footnote{See \ref{subsec:brief-intro-to-lojban} for a reminder on the meaning of "sumti"}.
\end{itemize}

Outside of the class, we can find an "if-clause".
This is the entrypoint of the file if it is executed directly through a terminal.
In that case, the Jbovlaste class is instantiated and the "get\_word\_pretty" function
is executed, with whatever word was given as an argument in the call.

\newpage

\section{Examples of usage}

This module can be used in two different ways: as a command-line interface (CLI) tool
from a terminal, or as a class for short scripts or other modules to use.

\begin{lstlisting}[caption=Jbovlaste module being used as a command-line interface tool]
$ python3 dictionary/jbovlaste.py "blanu"

blanu

Word Type: gismu
Definition: $x_{1}$ is blue [color adjective].
Glossary Words:
        - blue
\end{lstlisting}

\begin{lstlisting}[language=Python, caption=Jbovlaste class being used by a Python script]
>>> from dictionary import Jbovlaste
>>> jbovlaste_instance = Jbovlaste()
>>> jbovlaste_instance.get_word_struct("blanu")
{'definition': '$x_{1}$ is blue [color adjective].', 'glosswords': [{'word': 'blue', 'sense': None}], 'type': 'gismu'}
\end{lstlisting}